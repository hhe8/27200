\documentclass[11pt]{article}
\usepackage{amsmath,amssymb}
\usepackage{mathtools}
\usepackage[top=1in, bottom=1in, left=1in, right=1in]{geometry}
\title{Homework 6 due Friday 2/18/2015}
\author{Huimin He , section 1}
\begin{document}
\maketitle

%define macros
\def\bfor{\textbf{for}}
\def\bdo{\textbf{do}}
\def\bwhile{\textbf{while}}
\def\bend{\textbf{end}}
\def\bif{\textbf{if}}
\def\bthen{\textbf{then}}
\def\belse{\textbf{else}}
\def\breturn{\textbf{return}}
\def\bexit{\textbf{exit}}
\def\bprint{\textbf{print}}
%end macros

\begin{enumerate}
\item 6.6 (c) show that $ab\equiv xy \pmod{m}$.

Given $a \equiv x \pmod{m}$ and $b \equiv y \pmod{m}$, we have
\[ mk_1 = a-x\] and \[mk_2 = b-y\] where $k_1,k_2$ are integers.
\[mk_{2}a=a(b-y)=ab-ay\]
\[mk_{1}y=ay-xy\]
Adding the above two equations we get
\[m(k_{2}a+k_{1}y) = ab -xy\]
since $k_1,k_2,a,y$ are integers, $m$ divides $ab-xy$. So by definition
$ab \equiv xy \pmod{m}$ is proved.
\\

\item 6.8 Eclid's rounds Exercise 2.3 of the handout

let $B_i$,$R_i$, and $q_i$ be the variable $B,R,q$ after $i$ iterations.We have \[B_{i+2} = B_i - B_{i+1}q_{i+2}\]
By division theorem we know that \[0 \leq B_{i+2} < q_{i+2}\]

Divide both sides by $B_i$
So \[ \frac{B_{i+2}}{B_i} = \frac{B_i}{B_i} - q_{i+2}\frac{B_{i+1}}{B_i}\]
\[\frac{B_{i+2}}{B_i}(1+q_{i+2}) = 1\]
\[\frac{B_{i+2}}{B_i} = \frac{1}{1+q_{i+2}}\]
Since \[ 1 \leq q_{i+2}\] from $B_{i+1} < B{i}$
So \[\frac{B_{i+2}}{B_i} \leq \frac{1}{2}\] for all $i$ is proved. 

\item 6.9 compute $21^-1 \bmod{76}$

\textbf{part(a)}\\
We want find $x$ such that \[21x \equiv 1 \pmod{76}\]
We know \[76x \equiv 0 \pmod{76}\]
So \[76x-3\times21x \equiv 0-3 \pmod{76}\]
\[13x \equiv -3 \pmod{76}\]
substract this from the first equation
\[ 21x -2\times13x \equiv 1+2\times3 \pmod{76}\]
\[-5x \equiv 7 \pmod{76}\] substract this again from the equation above.
\[13x-2\times(-5x) \equiv -3+14 \pmod{76}\]
so \[3x \equiv 11 \pmod{76}\]
\[2x \equiv -18 \pmod{76}\]
\[x \equiv 29 \pmod{76}\]

\textbf{part(b)}\\
\begin{align*}
&gcd(228,63)\\ 
&=gcd(228-3\times63,63)\\
&=gcd(63,39)\\
&=gcd(39,63-39\times2)\\
&=gcd(39,-15)\\
&=gcd(9,-15)\\
&=gcd(-15+9\times2,9)\\
&=gcd(9,3)\\
&=gcd(6,3)\\
&=gcd(3,0)\\
&=3
\end{align*}
so \[3=228u+63v\] where $u,v$ are integers.
Divide both sides by 3 we have \[1=76u+21v\]
so \[ 76u \equiv 1 \pmod{21}\]
To calculate $u$, notice that
\[ 21u \equiv 0 \pmod{21}\] substract this we have 
\[76u-21u\times3 \equiv 1-0 \pmod{21}\]
\[13u \equiv 1 \pmod{21}\]
\[21u - 13u\times2 \equiv 0-2 \pmod{21}\]
so \[-5u \equiv -2 \pmod{21}\]
\[21u-5u\times4 \equiv 0-8 \pmod{21}\]
\[ u \equiv -8 \pmod{21}\]
\[ u \equiv 13 \pmod{21}\]
take $u=13$ and we can calculate that $v=-47$
\[3 = 228\times13+63\times(-47)\]

\item 6.10 Multiplicative inverse:pseudocode

We want to compute $ux \equiv 1 \pmod{m}$

Input: let $a= max(u,m)$,$b=min(u,m)$. If $u>m$ then $c=1,d=0$,otherwise $c=0,d=1$. Use euclid's algorithm to test if $gcd(a,b)=1$. (note if $u=m$, $gcd(a,b)=m=u$.)\\

Output: multiplicative inverse of $u$ mod m.

\textbf{Pseudocode}
\begin{tabbing}
mmm\= mm\= mm\= mm\= mm\= \kill

Test existence of multiplicative inverse\\
0 \> \bif\ $gcd(a,b) \neq 1$\\
1 \> \> \breturn\ $FALSE$\\
\\
Compute multiplicative inverse\\
2 \> Initialize: $A:=a$, $B:=b$, $C:=c$, $D:=d$, $T=C$, $R=0$, $q=0$\\
3 \> \bwhile\ $B \geq 1$ \bdo\\
4 \> \> $R: = (A \bmod B)$\\
5 \> \> $q: = (A-R)/B$\\
6 \> \> $C:=D, D:=T-qD, T:=C$\\ 
7 \> \> $A:=B, B:=R$\\
8 \> \bend\ (\bwhile\ )\\
9 \> \breturn\ $D$\\

\end{tabbing}

The running time of this algorithm is constant times the running time of euclids algorithm. Line 0 and 1 are added to test the exisitence of multiplicative inverse. Line 5 and 6 are added to compute the multiplicative inverse.  

\item 6.12 application of Fermat's little Theorem\\
From Fermat's little Theorem, since $gcd(7,101) = 1$. We have
\[ 7^{101-1} \equiv 1 \pmod{101}\]
so \[ (7^{10^2})^{10^7} \equiv 1^{10^7} \pmod{101}\]
\[7^{10^9} \equiv 1 \pmod{101}\]
so\[7^{10^9} \bmod 101 = 1\]

\item 6.14\\
Claim: Yes, Given $N,K$, $p,q$ can be calculated in polynomial time.

We know \[ N=pq\]
\[K=(p-1)(q-1)\]
so
\[p+q =N-k+1\]
\[pq=N\]
is a system of 2 equations with 2 unknows. We can solve this with formula.\\
let $x_1=p$,$x_2=q$, we can write \[x^2-(p+q)x+pq=0\]
so \[x^2-(N-K+1)x+N=0\]
so\[p,q = x_1,x_2 = \frac{N-K+1 \pm \sqrt{(N-k+1)^2-4N}}{2} \]





\item 6.16 Frank's algorithm time complexity\\
let $b$'s bit length be $n$, so $b < 2^n$. The algorithm terminates at $b^{\frac{1}{2}}$. Plug in $b = 2^n$.
\[ T(n) < 2^{\frac{n}{2}}\]
\[ T(n) = O(2^{\frac{n}{2}})\]
The algorithm finishes in exponential time.


\item 6.17 Ashwin and Ming secure scheme break

Ashwin picks primes $p,q$ and Ming picks primes $p,r$.
Since $n = pq$ and $m = pr$ are known from the public keys of Ashwin and Ming. We can use Euclid's algorithm to find $p$ which is $gcd(n,m)$ in polynomial time. 

\end{enumerate}

\pagebreak
Extra credit\\ reference to http://www.cut-the-knot.org/arithmetic/GcdLcmProperties.shtml
wan to show that \[lcm(gcd(n_1,m),gcd(n_2,m),...,gcd(n_k,m) = gcd(lcm(n_1,n_2,...,n_k),m)\]
Proof:
let $p$ be a prime divisor of $lcm(gcd(n_1,m),...gcd(n_k,m))$. let $a$ be the largest exponent such that $p^a|lcm(gcd(n_1,m),...,gcd(n_k,m)$. so $p^a$ divides at least one of $gcd(n_i,m) i =1,2,3,..,k$. let this number be $gcd(n_1,m)$. So $p^a$ is common divisor of $n_1$ and $m$.since $p^a|m$, it also divides $lcm(n_1,n_2,...,n_k)$. So $p^a$ is the common factor of $lcm(n_1,...,n_k)$, so $p^a|gcd(lcm(n_1,...n_k),M)$.


\end{document}
