\documentclass[11pt]{article}
\usepackage{amsmath,amssymb}
\usepackage{mathtools}
\usepackage[top=1in, bottom=1in, left=1in, right=1in]{geometry}
\title{Homework 9}
\author{Huimin He , section 1}
\begin{document}
\maketitle


\def\bfor{\textbf{for}}
\def\bdo{\textbf{do}}
\def\bwhile{\textbf{while}}
\def\bend{\textbf{end}}
\def\bif{\textbf{if}}
\def\bthen{\textbf{then}}
\def\belse{\textbf{else}}
\def\breturn{\textbf{return}}
\def\bexit{\textbf{exit}}
\def\bprint{\textbf{print}}


\begin{enumerate}

\item 9.1 HW (Knapsack decision problem NP-complete)\\
(a) Input: value parameter $M>0$, list of values $[v_1,...,v_n]$, list of weights $[w_1,...w_n]$, weight limit $W$. \\
Question: does a subset $ S \subseteq$ \{$1,...,n$\} exist s.t.
$ \sum\limits_{k \in S} v_k \geq M$ and 
$ \sum\limits_{k \in S} w_k \leq W$. \\

(b) witness: a subset $ S \subseteq$ \{$1,...,n$\} \\
Relevant facts:
$ \sum\limits_{k \in S} v_k \geq M$ and 
$ \sum\limits_{k \in S} w_k \leq W$. \\

c) to show that KNAP is NP - complete, we need to show that $KNAP \in NP$ and  construct the following reduction
\[ SUBS \preccurlyeq_p KNAP\] 
To reduce an instance of SUBS to KNAP, create the following KNAP problem let \[v_i = w_i = s_i\] where $s_i$ is the integer numbers in the subset sum problem.\\
 Subset sum problem:
 Instance: Non-negative integer numbers $s_1, s_2 , ... ,s_n, and t$\\
 Question: Does a subset of these numbers add up to $t$?

Denote $b$ to be the weight limit, $k$ to be the profit that we want to ask if the knapsack has value at least $k$.

contruction:\\
\[v_i = w_i = s_i\]
\[ t=b=k\]
Then the knapsack problem is reduced to the subsetsum problem because
\[ \sum_{i \in S} w_i \leq b\]
\[ \sum_{i \in S} v_i \geq k\]
$\iff$ 
\[ \sum_{i \in S} s_i = t\]

Yes answer to the subset sum problems gives the left part of the relationship and satisfies the instance for knapsack problem.


\item 9.2(NP-hard problems)\\
(a) B implies A so (a2) is correct. Because cook reductions needs an oracle
to solve membership in $L_2$ in order to solve membership in $L_1$ but there is no oracle in Karp reduction. 

(b) Integer Knapsack problem\\
Input: list of values $[v_1,...,v_n]$, list of weights $[w_1,...w_n]$, weight limit $W$. \\
Optimization Objective:
Find a subset $ S \subseteq$ \{$1,...,n$\} to 
maximize the $V_max = \sum\limits_{k \in S} v_k \geq M$ under the constraint that 
$ \sum\limits_{k \in S} w_k \leq W$. \\

Output: $V_max$

(c)Prove that integer knapsack is np-hard by cook-reduction from KNAP.\\
Denote the integer knapsack optimization problem as KNAP-opt. We only need to show that KNAP $\preccurlyeq_cook$ KNAP-opt in order to show Knap-opt is np-hard because KNAP is np-complete from HW 9.1.

Algorithm to solve KNAP-opt calling KNAP:
We can apply binary search to find the $V_max$.\\ 
Refer the KNAP algorithm to HW9.1\\
First choose value $M$ to be $V_total$,the sum of value list $[v_1,...,v_n]$ run KNAP\\
If the answer is yes, increase $M$ by $(upbound-M)/2$ and run KNAP again. Also update $lowerbound = M$\\
If the answer is no, decrease $M$ by $(M-lowerbound)/2$ and update $upperbound = M$ and run KNAP again\\ 

The oracle needs to be called at most $log_2{V_{total}}$ times.

\item 9.3 Hamiltonian path
To show that HAMILTON-PATH is NP-hard, we need to show that HAMILTONIAN $\preccurlyeq_{cook}$ HAMILTON-PATH.

Pseudocode: solve HAMILTONIAN USINGHAMILTON-PATH oracle.\\
Input: a graph $G$
Output: yes/no answer

\begin{tabbing}
mmm \= mm\= mm \= mm\= mm\= \kill

Initilization\\
0 \> add a new vertex $u$ to G\\
1 \> pick a vertex $v \in G$\\
3 \> connect all vertices that are connected to $v$ to $u$\\
4 \> \breturn\ HAMILTON-PATH on the new graph$G'$\\

\end{tabbing}

Correctness of this reduction: hamiltonian cycle can be converted to path in $G'$ by starting at $v$ until the cycle reaches a vetex $s$ befroe returning to $v$. Finish at $u$ instead of $v$.

Hamitonian path from $v$ to $u$ in $G'$ can be converted to a Hamiltonian cycle in $G$ by starting with $v$ and when it reached $s$ before $u$, return to $v$ instead.

\item 9.4 Metrix traveling salesman\\
(a) given a graph G described in problem. Is there a hamilton cycle with cost less than $k$?

witness: a hamilton cycle in $G$ with cost less than $k$. This can be verified easily by checking weather it is a hamilton cycle and adding up the total cost to see if it is less than $k$.

(b) reference to the text book page 1097\\
Show that HAMTONIAN $\preccurlyeq_p$ MTS.
Denote $G=(V,E)$ as an instance of HAMTONIAN. We need to construct an instance of MTS.
From the complete graph $G' = (V,E')$, $E'={(i,j):i,j \in V and i \neq j}$, define the cost function $c$ by \\ $c(i,j) = 0$ if $(i,j) \in E$\\  $c(i,j) =1$ if $(i,j) \not\in E$\\

Graph $G$ has a hamiltonian cycle if and only if graph $G'$ has a tour of cost at most $0$. Suppose that graph $G$ has a hamiltonian cycle $h$. Each edge in $h$ belongs to $E$ and has cost $0$. Suppose graph $G'$ has a tour $h'$ of cost at most 0. Because the cost s of the edge in $E'$ are $0,1$. So $h'$ has only edges in E. So $h'$ is a hamiltonian cycle in $G$.

\item 9.5 ( amortized analysis queue via stack)\\
(a)loop invariant: totle number of element in $S$ and $R$ does ont change. If we concatenate $R$ stack and reverse $S$ stack , the order of the elements does not change. The last statement makes sure that $S$ eventually has elements from $R$ in reverse order.


\item 9.6 (MAX 3 sat problem)\\
(a1) sameple space: $\{0,1\}^n$ since we have $n$ variables.\\
(a2) random variable: $X$ is the number of satified clauses. let $I_i$ be the value of the $i$th clause. So 
\[X = \sum_i I_i\]
So \[E(X) = E\sum_i I_i = \sum_i E(I_i)\]
\[ E(I_i) = 1 \times P(I_i =1 ) + 0 \times P(I_i=0) = 1 \times 7/8 + 0 \times 1/8\]
Where $P(I_i=1) = 1-(1/2)^3 = 7/8$\\

(b) prove that $m \geq 7s/8$ is tight\\
Consider the contrsuction of the clauses as following\\
we have $8n$ clauses where $n$ is positive integer. Pick first $7n$ randomly from all the variables. Then pick $1n$ clauses such that each clause is a negation of one of the first $7n$ clauses. By doing this, we make sure that the number of clauses that are correct is at most$7/8n$ because the $1n$ contradicts the other $7n$ clauses. 

(d1)
We can treat the process as geomteric process with $p(success) = 1/(s+1)$ when $X \geq 7s/8$. let $N$ denote the expected number of trials needed untill one success. $N \sim$ geometric distribution.
\[E(N) = 1/p(success) = (s+1)\]

(d2)
The randomized algorithm runs in polynomial expected time. It needs $ s+1 $ expected trials to achieve $ X \geq 7s/8$.  





\end{enumerate}
\end{document}
